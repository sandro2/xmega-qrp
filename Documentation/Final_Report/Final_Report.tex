\documentclass[a4paper,12pt]{article}
\linespread{1.5}
\usepackage[left=2cm,top=2cm,right=2cm,bottom=2cm]{geometry}

%\usepackage{palatino}

\usepackage{xypic}
\usepackage{natbib}
\bibliographystyle{plainnat}
\bibpunct{[}{]}{;}{s}{,}{,}

\usepackage[header,page,titletoc]{appendix}
\renewcommand{\appendixname}{Appendix}
%\renewcommand{\appendixtocname}{List of appendices}

\usepackage[colorlinks=true,linkcolor=black,citecolor=black,filecolor=black,menucolor=black,urlcolor=black]{hyperref}
\usepackage{graphicx}
\usepackage{amsmath}
\usepackage{pdfpages}
\usepackage{wrapfig}
\usepackage{fancyhdr}
\usepackage{pict2e}
\setlength{\headheight}{15.2pt}

\pagestyle{fancy}
\usepackage{cite}
%
% fix citations to be IEEE style
\def\citepunct{], [}
\def\citedash{]--[}

\newcommand{\nonumsection}[1]{
\section{#1}
%\addcontentsline{toc}{section}{#1}
}

%Used to change "Abstract" to "Executive Summary"

% Paragraph Settings
\setlength{\parindent}{0pt}
\setlength{\parskip}{5pt}

\begin{document}
\thispagestyle{empty}
\vspace*{\fill}
\includegraphics[width=10cm]{./UofAlogo.pdf}\\
\noindent
\textsc{
\textsc{School of Electrical \& Electronic Engineering}\\
Adelaide, South Australia, 5005\\ \\
}
\noindent
\Large{\textbf{
ELEC ENG 4039A/B \\
Honours Project\\
	}}
	\Large{
		A Radio Relay System for Remote Sensors in the Antarctic \\
	}
	\small{\textbf{Supervisors: Dr. Chris Coleman, Dr Said Al-Sarawi}}
	\ \\
	\ \\
	\Large{\textbf{
		Final Report \\
	}}
	\ \\
	\small{\textbf{
		Written by: \\}
		Mark Jessop \\
		1163807
	}
	\ \\
	\ \\
	Date Submitted: October 22, 2010 \\
	Signature of Supervisor: \\
 %\end{center}
 \vspace*{\fill}

\newpage
 \thispagestyle{empty}
 \vspace*{\fill}
\begin{abstract}
\noindent
A common problem with remote sensor systems is the retrieval of data. Satellite-based systems are expensive, as is travelling to the sensor. HF propagation provides an inexpensive alternative. Radio signals below 30MHz can easily bounce off the ionosphere, travelling thousands of kilometres using only a few watts of transmit power.
Based around an Atmel XMega Micro-Controller and using Direct Digital Synthesis techniques, this project aims to provide a reliable low power HF telemetry system, usable in a variety of remote telemetry applications. 
By making use of the XMega's power-save modes and using high-efficiency RF amplifiers, power consumption is minimised, allowing months of operation from battery power.
\end{abstract}
\vspace*{\fill}
\newpage
\tableofcontents
\newpage

\section{Introduction}
Many scientific experiments require data logging over a long time period, and in remote locations. To further research progress, it is desirable for some of the collected data to be accessible throughout the run of the experiment.

A number of solutions exist to obtain data from a remote sensor unit. The first is to physically access the remote sensor station, and copy data from whatever storage may exist. For very remote sensor units visiting the site may be impractical or too costly, so some form of wireless communication is often used. 

Satellite data transmission is a reliable way of retrieving data, but comes at a high cost. For example, using the Iridium Satellite constellation to transmit data would cost approximately USD\$2.50 per minute, with a 2400 baud data rate\citep{ref:iridium}.

The other option is to other methods of radio communication. For reasonably short ($<$50km) distances VHF or UHF telemetry can be used. For example, the Australian Bureau of Meteorology uses a network of VHF transmitters\citep{ref:bomtx} and repeaters to obtain data from weather stations around Adelaide. VHF and UHF can work for longer distances ($>$50km), but with less reliability. Transmission at these distances relies on tropospheric ducting, a phenomenon that is not always present. To obtain high reliability long distance transmission, we must move further down the electromagnetic spectrum, to the HF (`high frequency') band.

\subsubsection*{The Ionosphere}

Radio waves in the HF band propagate mainly by two means: ground-wave and sky-wave. Ground-wave, as the name suggests, travel along the surface of the earth. Ground-wave signals are attenuated as they travel along the earth's surface, limiting their distance. Sky-waves however, refract off a charged region of the earth's atmosphere called the ionosphere, and can travel extremely long distances.

The ionosphere is a complex and ever-changing layer (actually multiple layers) of ionized particles that surrounds our planet. It consists of multiple layers , given the letters D, E and F. The D-layer, which is the lowest and is strongest during the day, attenuates (absorbs) RF energy below a certain frequency. The E and F layers, however, \textit{reflect} RF energy below a certain \textit{critical frequency}, $f_c$. This property enables HF radio signals to be `bounced' off the ionosphere, even multiple times, to communicate over long distances. Losses from the D-layer, and the critical frequencies of the E and F layers determine the Lowest Usable Frequency (LUF) and Maximum Usable Frequency (MUF) for long distance communication. Ionospheric prediction services, such as the one provided by the Australian government \citep{ref:bom}, provide tools to predict these frequencies, and hence determine the optimal frequency for reliable communication.

Another consideration is the angle at which a radio wave interacts with the ionosphere. If a radio signal is aimed at the horizon it will hit the ionosphere at a shallow angle, resulting in very long propagation distances, possibly thousands of kilometres. For shorter distances the signal is aimed close to vertical, allowing much shorter hops, perhaps a few hundred kilometres. This method of transmission is called `Near Vertical-Incidence Skywave' (NVIS) and is the primary method this project will use to transmit data.

\subsubsection*{Project Applications \& Aim}


\begin{quote}
\textit{
Design and build a low power short wave data transmitter that can translate low rate data into a modulated short wave output with sufficient power to transmit the data over several hundred kilometres on an Antarctic communication path.}
- Original Project Aim
\end{quote}


This project was originally intended to be used by a researcher stationed in Antarctica, hence the project name. Throughout the course of the project, lack of information from the researcher prompted a widening of the project's scope to include other telemetry applications. Examples include:
\begin{itemize}
\item Use in the Australian outback for energy research.
\item Use on remote islands, for weather monitoring.
\item As a telemetry system for a high-altitude weather balloon. 
\end{itemize}

At it's core, the purpose of transmitter is to read in data from a number of sources, buffer the data, and then transmit it using some form of HF radio modulation. The exact form of HF modulation was not defined, to allow experimentation with a number of different modes. To enable high power testing without many licensing issues, it was decided that the project would target the 80m (3.5MHz) and 40m (7MHz) amateur radio bands. This also had the added benefit of having a large number of radio operators available to report on signal reception.

To keep the spirit of the original application it was decided to keep the constraints which Antarctic operation required. These were:
\begin{enumerate}
\item Must be able to operate at extremely cold temperatures ($<-30^\circ$C), such as those commonly experienced in Antarctica.
\item Must be very energy efficient, to enable operation from battery power for long periods.
\end{enumerate}

These constraints formed the guidelines for the design and construction of a prototype transmitter, which was successfully tested and then flown as a payload on a high-altitude weather balloon.

\newpage
\section{Hardware}

\subsection{CPU}
\begin{wrapfigure}{r}{0.3\textwidth}
  \begin{center}
    \includegraphics[width=0.28\textwidth]{images/xmega100.jpg}
  \end{center}
  \caption{XMega100 Breakout Board}
  \label{fig:xmega100}
\end{wrapfigure}
The main CPU of the HF transmitter is an Atmel AT-XMega128A1, an 8-bit micro-controller running at clock speeds up to 32MHz. It has numerous peripherals, such as 12-bit ADCs, a 12-bit DAC, and many I/O lines. A particularly interesting feature is the ability to wire SD-RAM into three of the I/O ports and have the extra memory appear at the end of the XMega's memory space. This is called the `External Bus Interface' (EBI) and allows the buffering of large portions of data with little external circuitry.

The XMega was chosen over other similar micro-controllers (such as the MSP430) primarily due to the open-source and cross-platform compiler tools available. Software development was primarily carried out on an Apple Macbook, and having a compiler and programmer working without rebooting into Microsoft Windows proved very useful.

For the purposes of easy development, two breakout boards were purchased. The first, a Sparkfun XMega100 breakout board (Figure \ref{fig:xmega100}), simply gives access to the I/O pins on the IC, and was bought as a means to evaluate the micro-controller as a viable platform. The second, an Atmel XPlain development board, has buttons, LEDs, and a number of extra peripherals, such as 8MB of SDRAM and 8MB of NAND-Flash memory. This board was used for most development work, as the buttons and LEDs proved extremely useful for debugging code.

\begin{wrapfigure}{l}{0.4\textwidth}
  \begin{center}
    \includegraphics[width=0.4\textwidth]{images/xplain.jpg}
  \end{center}
  \caption{XPlain Development Board}
  \label{fig:xplain}
\end{wrapfigure}

When running at 32MHz with most peripherals disabled, the XMega was measured to draw 20mA of current at 3.3V. Running at 2MHz, the current draw was measured at 2mA. The XMega also has a number of power-save modes, where an ultra-low-power 32KHz oscillator is used. The data-sheet states these modes draw between 0.1 and $3\mu A$ depending on what functions are enabled, but this has not been tested. From this information, we can see that the XMega's power requirements are extremely minimal, especially when running at low clock speeds.

To see if the XMega128A1 would satisfy the cold-climate operation requirement, the Sparkfun breakout board was subjected to low temperature testing. Using frozen CO$_2$ (dry-ice), the temperature of the XMega was lowered to $-54^\circ$C. The internal 32MHz oscillator drifted upwards to 33MHz, with the XMega still continuing to function. Further details of the test process appear in  Appendix \ref{xmegadryice}.


\subsection{Signal Generator}
\begin{wrapfigure}{r}{0.3\textwidth}
  \begin{center}
    \includegraphics[width=0.28\textwidth]{images/ad9835.jpg}
  \end{center}
  \caption{AD9835 Breakout Board}
  \label{fig:ad9835}
\end{wrapfigure}
The start of the transmitters RF section begins with the signal generator. Two signal generators were experimented with for this project, the Analog Devices AD9834 and AD9835. Both devices are programmable Direct Digital Synthesis (DDS) signal generators, able to produce sine wave output between 1Hz and 25MHz. 

Inside each device is a 10-bit 50Msps DAC, clocked using an external 50MHz oscillator. Serial Peripheral Interface (SPI) programmed control registers allow the programming of two different frequencies. These can be selected between using either control bits or a dedicated input pin, allowing FSK modulation. The registers can be reprogrammed approximately 7500 times per second, allowing MFSK modulation with high symbol rates. Phase registers (2 in the AD9834, 4 in the AD9835) permit BPSK and QPSK modulation, though this was not used in the project.

The AD9834 is a lower power device, drawing 8mA at 5V, while the AD9835 can draw up to 40mA. In practice, it was found that when the current draw of the required master oscillator was included, the power requirements of both devices (with supporting circuitry) were very similar (~35mA @ 5V).

The AD9835 was purchased on a breakout board from Sparkfun prior to the project, which started the idea of using a DDS to power the RF section. The AD9834 was purchased as a bare IC, and a breakout PCB was designed and constructed for it. Schematics and PCB designs for this board appear in Appendix \ref{ad9834_breakout}. The breakout was originally designed to run at 3.3v, the same voltage level as the micro-controller. However, because the board ended up using a 5V master oscillator (3.3v models were not available at the time), it had to be operated entirely at 5V. Even running at 5V the AD9834 accepts 3.3v logic levels from the XMega, unlike the 5V AD9835 which requires logic level conversion.

Being a DAC-based signal generator, the output waveform is not a pure sine. Sampling theory tells us that at 10MHz we would only 5 samples per cycle, and at 25MHz only 2 samples per cycle. At the target frequency of 7MHz, clock feedthrough (50MHz) was measured at -20dB below the target output. In Australia, spurious emissions should be -30dB below the fundamental frequency, so a reasonably sharp low-pass filter was designed to drop the clock feedthrough down by about -50dB. Use of a tuned antenna will may also attenuate the clock feedthrough enough to meet legal limits, but this could not be tested easily.

\subsection{Pre-Amplifier}

\subsection{Class E Power Amplifier}


\section{Software}

\subsection{Hardware Libraries}

\section{Testing}

\section{Project Management}

\section{References}
\renewcommand*{\refname}{\vspace*{-12mm}}
\begin{thebibliography}{99}
\bibitem{ref:iridium}
Iridium Communications Inc. ``Iridium Satellite Call Plans"" \url{http://www.iridiumphones.com.au/Call\%20Plan\%20Brochure\%20-\%20Post-paid.pdf}, 2010

\bibitem{ref:bomtx}
ACMA Register of Radio-communications Licenses, Met Bureau Site near Corkscrew Rd Montacute \url{http://web.acma.gov.au/pls/radcom/assignment\_search.lookup?pACCESS\_ID=1320447&pDEVICE\_ID=1316356}

\bibitem{ref:bom}
Australian Bureau of Meteorology, ``IPS Online HF Network Frequency Selection Tool" \url{http://www.ips.gov.au/HF\_Systems/7/1/10}, 2010 %[Mar. 14, 2010]



\end{thebibliography}


\newpage
\begin{appendices}
\section{XMega Dry Ice Testing}
\label{xmegadryice}
To test the AT-XMEGA128A1 breakout board, the chip was programmed to output the internal clock signal on a I/O pin, and one of the UARTs was programmed to continually send out ``Hello World" at 9600 baud. A LED was also attached to the board, and the chip programmed to flash it at 1Hz. The breakout board assembly was wrapped in bubble wrap (for insulation) with a ribbon cable exiting the insulation to carry the data lines. The insulated board was then placed in a small foam Eski containing approximately 3KG of dry ice. Over the course of an hour, the chip cooled down to -49$^\circ$C, where it stayed for approximately 20 minutes. After shuffling the dry ice slightly, the chip cooled down a further 5$^\circ$C, to -54$^\circ$C at which point the test was aborted. Throughout the test the clock frequency and chip temperature were measured, to produce the plot in Figure \ref{32mhzrc}.


\begin{figure}[h!]
\begin{center}
\includegraphics[width=13cm]{images/32MHzRC.pdf}
\caption{AT-XMEGA128A1 RC Clock Drift}
\label{32mhzrc}
\end{center}
\end{figure}

\begin{figure}[h!]
\begin{center}
\includegraphics[width=13cm]{images/xmega_4.jpg}
\caption{AT-XMEGA128A1 Breakout Board Wrapped in Insulation}
\label{xmega_4}
\end{center}
\end{figure}

\newpage
\begin{figure}[h!]
\begin{center}
\includegraphics[width=13cm]{images/xmega_2.jpg}
\caption{AT-XMEGA128A1 Breakout Board Warming up after testing.}
\label{xmega_2}
\end{center}
\end{figure}

\begin{figure}[h!]
\begin{center}
\includegraphics[width=11cm]{images/xmega_3.jpg}
\caption{Ice forming on the XMEGA's pin headers after testing.}
\label{xmega_3}
\end{center}
\end{figure}


\section{Software}
All software for this project has been released under the GPLv3 license. 

\subsection{The Arduino Project and the XMega}
The Arduino is an
\begin{quote}
open-source electronics prototyping platform based on flexible, easy-to-use hardware and software.\footnote{http://arduino.cc/}
\end{quote}
Since many users of the Arduino haven't had much experience in programming, many libraries have been created to fulfil various needs. It's fairly common to begin working on a bit of code to drive some chip, to find that someone has already written a library a few months previously. To quickly get sections of the XMega's codebase working, it was decided to port certain Arduino libraries to the XMega platform. 

Arduino's IDE uses the `wiring' programming language. Wiring is a `C-like' language, following most of C's syntax, to the point that the Arduino IDE uses a collection of C++ libraries to `convert' wiring to C++ for compilation. While these libraries could be ported to the XMega allowing usage of the Arduino IDE for this project, this would have required a lot of work and been out of scope. Instead, the Arduino libraries which were needed (OneWire, DallasTemperature, TinyGPS) were individually ported. This involved replacing various Arduino-specific function calls with generic AVR-C calls, and some other minor code changes. 


\section{Schematics \& PCBs}
\subsection{AD9834 Breakout Board}
\label{ad9834_breakout}
\begin{figure}[h!]
\begin{center}
\includegraphics[width=12cm]{images/AD9834_Schem.pdf}
\caption{AD9834 Breakout Board Schematic}
\label{ad9834_schem}
\end{center}
\end{figure}
\begin{figure}[h!]
\begin{center}
\includegraphics[width=12cm]{images/AD9834_PCB.pdf}
\caption{AD9834 Breakout Board PCB Artwork}
\label{ad9834_PCB}
\end{center}
\end{figure}

\end{appendices}

\end{document}