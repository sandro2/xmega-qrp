\documentclass[a4paper,12pt]{article}
\linespread{1.3}
\usepackage[left=3cm,top=2cm,right=3cm,bottom=2cm]{geometry}

%\usepackage{palatino}

\usepackage{xypic}
\usepackage{natbib}
\bibliographystyle{plainnat}
\bibpunct{[}{]}{;}{s}{,}{,}

\usepackage[header,page,titletoc]{appendix}
\renewcommand{\appendixname}{Appendix}
%\renewcommand{\appendixtocname}{List of appendices}

\usepackage[colorlinks=true,linkcolor=black,citecolor=black,filecolor=black,menucolor=black,urlcolor=black]{hyperref}
\usepackage{graphicx}
\usepackage{amsmath}
\usepackage{pdfpages}
\usepackage{fancyhdr,lastpage}
\usepackage{wrapfig}
\usepackage{float}
\usepackage{pict2e}
\setlength{\headheight}{15.2pt}

% Headings \& Footers
\pagestyle{fancy}
%\lhead{\scriptsize{Advanced Telecommunications}}
%\rhead{\scriptsize{\today}}
%%\rfoot{\scriptsize{A1-ProjectPlan.tex}}
\fancyfoot[C]{\scriptsize{Page \thepage\ of \pageref{LastPage}}}


\usepackage{cite}
\usepackage{multicol}

% Font Change
\usepackage{palatino}

% Stuff to make verbatim use small fonts.
\makeatletter 
\g@addto@macro\@verbatim\small 
\makeatother

%
% fix citations to be IEEE style
\def\citepunct{], [}
\def\citedash{]--[}

\newcommand{\nonumsection}[1]{
\section{#1}
%\addcontentsline{toc}{section}{#1}
}

%Used to change "Abstract" to "Executive Summary"
\renewcommand{\abstractname}{Executive Summary}

\newcommand{\tm}{$^{\mbox{\tiny{TM}}}$}

% Paragraph Settings
\setlength{\parindent}{0pt}
\setlength{\parskip}{5pt}

\begin{document}
\thispagestyle{empty}
\vspace*{\fill}
\includegraphics[width=10cm]{./UofAlogo.pdf}\\
\noindent
\textsc{
%\textsc{School of Electrical \& Electronic Engineering}\\
%Adelaide, South Australia, 5005\\ \\
}
\noindent
\Large{\textbf{
PHIL\_2045  \\
Professional Ethics\\
}}

	\ \\
	Essay 1\\
	\Large{\textbf{
			When is Whistle-blowing Acceptable? \\
	}}
	\ \\
	\small{\textbf{
		Written by \\}
		Mark Jessop - 1163807\\
	}
	\ \\
	\ \\
 %\end{center}
 \vspace*{\fill}

\newpage

\section*{Introduction}
Whistle-blowing is the act of raising concern about some form of wrongdoing in an organisation, through informing some higher authority or the public at large. While the name actually derives from English police blowing whistles upon noticing a crime, Sissela Bok makes the parallel with sports umpires blowing their whistle when calling a foul (Bok 1980). In most sports clear rules exist to determine if some action constitutes a foul. For whistle-blowing, no clear-cut rules exist. Various people have made attempts to define a set of rules to determine if whistle-blowing is acceptable. In this essay I aim to examine some of these attempts, and give examples as to when they might break down.

\subsection*{Sometimes things get complicated}
The most clear-cut examples of when whistle-blowing is the right action usually involve some organisation performing some clandestine illegal action which could, or already does, harm the public. These aren't the kinds of situations that require rules - It's obvious that something bad is going on, and that something should be done about it.

Unfortunately, that which is legal is not always moral, and that which is illegal is not always immoral. The potential whistle-blower must now make their own moral judgement on whether the situation is severe enough to warrant whistle-blowing. Most professional organisations have a code of ethics with some condition involving disclosure of dangers to the public. Point one of the IEEE's (Institute of Electrical and Electronics Engineers) code of ethics \footnote{\url{http://www.ieee.org/membership_services/membership/ethics_code.html}} reads:
\begin{quote}
1. [Members will] accept responsibility in making decisions consistent with the safety, health and welfare of the public, and to disclose promptly factors that might endanger the public or the environment;
\end{quote}
It is an IEEE member's responsibility to promptly disclose dangers to the public, but not necessarily to the public - most likely to a supervisor. Richard De George mentions this in his five conditions which make whistle-blowing obligatory:
\begin{quote}
\emph{\begin{enumerate}
\item ``Serious and considerable harm to the public" is involved;
\item one reports the harm and expresses moral concern to one's immediate supervisor;
\item one exhausts all other channels within the corporation;
\item one has available ``documented evidence which could convince a reasonable, impartial observer that one's view of the situation is correct"; and
\item one has ``good reasons to believe that by going public the necessary changes will be brought about" to prevent the harm.
\end{enumerate}}
\end{quote}

Here is our first example of an attempt to define a set of rules for whistle-blowing - let's take a look at each of them in more detail.

\subsubsection*{``Serious and considerable harm to the public"}

The first condition given is probably the most common reason for whistle-blowing. However, not all whistle-blowing cases involve harm to the public. Instead, it could be harm to the environment, or an economy. 




\section*{Private First Class Bradley E. Manning}
On the 5th of April, 2010, the well-known whistle-blowing website `Wikileaks' released classified video \footnote{http://www.collateralmurder.com/} showing three US Army airstrikes in Baghdad, Iraq. While the video shows Apache helicopter crews shooting suspected Iraqi insurgents later found with RPGs, it also shows the murder of two Reuters news employees and the wounding of women and children. 

In May 2010 Private First Class Bradley E. Manning, an army intelligence analyst, was arrested and later charged with two offences relating to mis-handling and unauthorised release of classified information. Manning is currently being held in solitary confinement at the Quantico military facility in Virginia, USA. The arrest came after Adrian Lamo, a former computer hacker with whom Manning confided in, went to the US Army with conversation logs showing Manning admitting to releasing classified data to Wikileaks. 

This case, which is still ongoing, is a very interesting example of a now high-profile whistle-blower. Manning's actions have provoked worldwide media coverage of an event that would have otherwise gone un-noticed in the noise of the Iraq War. 

\subsection*{Unprecedented Access}
\begin{quote}
\emph{�If you had unprecedented access to classified networks 14 hours a day 7 days a week for 8+ months, what would you do?�}\\
- Bradley Manning, in an instant message to Adrian Lamo\footnote{http://www.wired.com/threatlevel/2010/06/wikileaks-chat/}
\end{quote}

Bradley Manning was an army intelligence analyst, and as part of his job, had access to a large amount of classified material. 



- Disillusionment with army
- Tie into De Georges \#2 rule

\begin{quote}
\emph{i immediately took that information and *ran* to the officer to explain what was going on� he didn�t want to hear any of it� he told me to shut up and explain how we could assist the FPs in finding *MORE* detainees�}\\
- Bradley Manning, in an instant message to Adrian Lamo\\
\end{quote}


Over the course of his stationing, he managed to smuggle out classified material on re-writable CDs and send it via secure channels to Julian Assange, one of the directors of Wikileaks.


\begin{quote}
\emph{(1:10:38 PM) bradass87: its open diplomacy... world-wide anarchy in CSV
format... its Climategate with a global scope, and breathtaking depth...
its beautiful, and horrifying...\\
(1:11:54 PM) bradass87: and... its important that it gets out... i feel,
for some bizarre reason\\
(1:12:02 PM) bradass87: it might actually change something\\
(1:13:10 PM) bradass87: i just... dont wish to be a part of it... at least
not now... im not ready... i wouldn't mind going to prison for the rest of
my life, or being executed so much, if it wasn't for the possibility of
having pictures of me... plastered all over the world press... [...]\\
}
- Bradley Manning, in an instant message to Adrian Lamo
\end{quote}

\section*{References}
Bok, S (1980) ``Whistleblowing and Professional Responsibility" \\
\\
IEEE, \emph{Code of Ethics},\\
\url{http://www.ieee.org/membership_services/membership/ethics_code.html}\\
\\
BBC News. (June 7 2010) \emph{US intelligence analyst arrested over security leaks},\\
\url{http://www.bbc.co.uk/news/10254072}\\
\\
Wired News. (June 6 2010) \emph{U.S. Intelligence Analyst Arrested in Wikileaks Video Probe}\\
\url{http://www.wired.com/threatlevel/2010/06/leak/}\\
\\
ABC News Australia. (July 7 2010) \emph{US soldier charged over Apache Wikileaks video}\\
\url{http://www.abc.net.au/news/stories/2010/07/07/2946534.htm?section=world}\\
\\
Wired News. (June 10 2010) \emph{�I Can�t Believe What I�m Confessing to You�: The Wikileaks Chats}\\
\url{http://www.wired.com/threatlevel/2010/06/wikileaks-chat/}\\
\\
BoingBoing. (June 19 2010) \emph{Wikileaks: a somewhat less redacted version of the Lamo/Manning logs}\\
\url{http://www.boingboing.net/2010/06/19/wikileaks-a-somewhat.html}

\end{document}
